% !TEX root = paper.tex

%% ***************************************************************************
%% My paper
%%
%% Authors: Emmett Brown, Marty McFly, Biff Tannen
%%
%% NOTE: this file will not compile until you called the script
%% generate-preamble.php once. See the file Readme.md to understand what
%% to do.
%%
%% This paper is an instance of the PaperShell template. For more
%% information, please visit https://github.com/sylvainhalle/PaperShell
%% ***************************************************************************
%% ---------------------------
%% Author preamble. The contents of this file change depending on the
%% paper class you choose.
%% ---------------------------
  
%%%%%%%%%%%%%%%%%%%%%%%%%%%%%%%%%%%%%%%%%%%%%%%%%%%%%%%%%%%%%%%%%%%%%%%%%%
%% This file was autogenerated by PaperShell v2.6.1 on 2023-04-18 07:28:33
%% https://github.com/sylvainhalle/PaperShell
%% DO NOT EDIT!
%%%%%%%%%%%%%%%%%%%%%%%%%%%%%%%%%%%%%%%%%%%%%%%%%%%%%%%%%%%%%%%%%%%%%%%%%%
\documentclass[journal]{IEEEtran}
%FOO
% Usual packages
\usepackage[utf8]{inputenc}      % UTF-8 input encoding
\usepackage[T1]{fontenc}         % Type1 fonts
\usepackage[english]{babel}      % Hyphenation
\usepackage{graphicx}            % Import graphics
\usepackage{cite}                % Better handling of citations
\usepackage[scaled]{helvet} % Scale Helvetica
\usepackage[bookmarks=false]{hyperref}            % Better handling of references in PDFs
\usepackage{comment}             % To comment out blocks of text

\usepackage{amssymb}
\usepackage{csvsimple}


% Title
\title{An Exploration of Ant Colony Optimization in the Traveling Salesperson Problem}

% Authors and affiliations
\author{%
\IEEEauthorblockN{{Braden} {Webb}%
}
\\
\IEEEauthorblockA{%
\textit{Mathematics}\\
%
}

\IEEEauthorblockN{{Caden} {Brinkman}%
}
\\
\IEEEauthorblockA{%
\textit{Computer Engineering}\\
%
}

\IEEEauthorblockN{{Gavin} {Bolin}%
}
\\
\IEEEauthorblockA{%
\textit{Computer Science}\\
%
}

%
}

%% ---------------------------
%% If you have other packages or command definitions you'd like to
%% include, write them there
%% ---------------------------
\usepackage{amsmath,amsfonts,amssymb}

%% Note: the subfig package is incompatible with the "unfixed" version
%% of the LIPICS class. It works here only because PaperShell includes
%% a fixed version --which is not the official version.
\usepackage{subfig}

%% We put one here: this is a dummy command for LaTeX, used to
%% tell Aspell to skip checking the spelling of what's inside
%% Usage: I like the word \nospellcheck{kwyjibo}
\newcommand{\nospellcheck}[1]{#1}

%% --------------------------
%% Personalized todo notes
%% --------------------------
\usepackage{xcolor}
\usepackage{todonotes}
%\newcommand{\todosylvain}[1]{\todo[inline,caption={},color=cyan]{\sf\small \textbf{@Sylvain:} #1}}
%\newcommand{\todoalex}[1]{\todo[inline,caption={},color=pink]{\sf\small \textbf{@Alex:} #1}}
%\newcommand{\todosebastien}[1]{\todo[inline,caption={},color=lime]{\sf\small \textbf{@Sébastien:} #1}}
%\newcommand{\todomichael}[1]{\todo[inline,caption={},color=lime]{\sf\small \textbf{@Michaël:} #1}}
%\newcommand{\todotous}[1]{\todo[inline,caption={},color=yellow]{\sf\small #1}}
%\newcommand{\todoedmond}[1]{\todo[inline,caption={},color=lime]{\sf\small \textbf{@Edmond:} #1}}

%% --------------------------
%% Placeholder for figure
%% --------------------------
\newcommand{\imagevide}{\framebox(200,200){IMAGE}}

%% --------------------------
%% Alter some LaTeX defaults for better treatment of figures:
%% See p.105 of "TeX Unbound" for suggested values.
%% See pp. 199-200 of Lamport's "LaTeX" book for details.
%% --------------------------
%% Parameters for all pages
\renewcommand{\topfraction}{0.9}	% max fraction of floats at top
\renewcommand{\bottomfraction}{0.8}	% max fraction of floats at bottom

%% Parameters for TEXT pages (not float pages)
\setcounter{topnumber}{2}
\setcounter{bottomnumber}{2}
\setcounter{totalnumber}{4}                 % 2 may work better
\setcounter{dbltopnumber}{2}                % for 2-column pages
\renewcommand{\dbltopfraction}{0.9}         % fit big float above 2-col. text
\renewcommand{\textfraction}{0.07}          % allow minimal text w. figs

% Parameters for FLOAT pages (not text pages)
\renewcommand{\floatpagefraction}{0.7}      % require fuller float pages
% N.B.: floatpagefraction MUST be less than topfraction !!
\renewcommand{\dblfloatpagefraction}{0.7}   %require fuller float pages

%\newtheorem{example}{Example}

\usepackage{newtxtext,newtxmath} % Times with math support. Must be placed here to avoid clash if amsthm is loaded in includes
\usepackage{microtype}       % Better handling of typo
%% Default path for graphics
\graphicspath{{fig/}}

\begin{document}

\maketitle
\begin{abstract}
%% ----------------------
%% Write your abstract here. Do not enclose it in an "abstract"
%% environment.
%% ----------------------
The NP-complete traveling salesperson problem (TSP) has been studied for decades by the
computer science community. It's combinatorial search space makes it difficult to
solve in a reasonable amount of time for any problem with more than a few cities.
Due to its simple nature, however, it has become a common benchmark and testing ground
for many new algorithms and optimization techniques, including the several variations
of the ant colony optimization (ACO) algorithm. For our final project, we implemented a 
simple version of the Max-Min Ant System algorithm, discuss its computational 
complexity, and analyze its performance empirically. We find that while our 
implementation could be improved in many ways and is dependent on the choice
of hyperparameters, it can nevertheless outperform both greedy and branch-and-Bound
algorithms on the asymmetric TSP.

% The traveling salesman problem has been the main focus of computer science researchers, economists and businessmen. If optimally implemented, the traveling salesman problem can result in decreased travel time and decreased loop times. This will allow the public to enjoy more time on vacation instead of using that time traveling and will allow businesses to minimize costs of daily routes traveled throughout the world. In our implementation of this famous algorithm we mimicked the routing of ants to determine where the optimal route will be. 
%% :folding=explicit:wrap=soft:mode=latex:

\end{abstract}

% Fixing bug in the definition of \markboth in IEEEtran class
% See http://tex.stackexchange.com/a/88864
\makeatletter
\let\l@ENGLISH\l@english
\makeatother


%% ---------------------------
%% If you wish to include additional packages, define new environments or
%% new commands, put them in the file includes.tex
%%
%% Write your abstract in the file abstract.tex.
%% ---------------------------

%% ---------------------------
%% Introduction
%% ---------------------------
\section{Introduction} %% {{{

We chose to implement the ant colony algorithm because its inherently stochastic
nature felt different from many of the other algorithms that we've worked with
in this class. Although genetic algorithm approaches are also probabilistic, ACO
algorithsm differ because they are multi-agent algorithms that introduce randomness
at each decision point in an agent's path, rather than at the stages of mutation and
crossover in a genetic algorithm. This makes ACO algorithms quite difficult to analyze,
but also makes them more flexible and adaptable to different problem domains.

%% }}} --- Section

%% ---------------------------
%% A section
%% ---------------------------
\section{Algorithm Explanation} %% {{{

As baseline algorithms for comparison, we used the ``default'' algorithm, which
finds a valid random solution to the TSP as provided in the original boilerplate
code for project 5, and a greedy algorithm. We also used the best branch-and-Bound
algorithm from project 5 as a baseline for comparison. Within this report, we
will not discuss the default algorithm or the branch-and-Bound algorithm in detail,
but we will discuss the greedy algorithm and the ACO algorithm below---particularly
since the greedy algorithm forms the first step of the ACO algorithm.

\subsection*{Greedy}
Our greedy algorithm simply takes the smallest edge cost branching out of the
current node we are at until all nodes are visited and we make a complete loop.
The starting city is chosen randomly, and if the path is not valid, we choose
a new starting city, looping through all possible starting cities until a
solution is found. While we eventually came to appreciate the quality of most
of these solutions (especially considering how quickly the algorithm runs), they
are almost never optimal.

\subsection*{Ant Colony Optimization Algorithm (ACO)}
We originally became interested in this algorithm when we stumbled across the
work of \cite{Skinderowicz_2022} from 2022. We thought it would be interesting
to implement their state-of-the-art ``Focused ACO'' algorithm that they propose;
however, we quickly realized that even the simpler versions on which it was based
were quite complex. We ended up only implementing one of the simpler versions
that they used as a baseline, the Max-Min Ant System (MMAS) algorithm. MMAS
was proposed by \cite{Sttzle2000MAXMINAS} in 2000, and is one of the most popular
variations of ACO algorithms.

The ant colony algorithm \cite{Skinderowicz_2022} mimics ants and their use of 
pheromones in order to locate potential food sources. This is done by detecting 
the concentration of each pheromone for each path and choosing based on 
probability the path with more pheromone. The pheromone on each edge decreases 
a little bit at each iteration, simulating the evaporation of pheromone that
occurs in nature.
Our implementation follows 5 steps:

\begin{enumerate}
  \item Initialize a population of $n$ ants
  \item Allow the ants to choose paths with more preference on shorter paths and
  with more preference on paths with a higher concentration of pheromones
  \item For ants that achieved a valid route, choose the one that had the lowest 
  cost \cite{Sttzle2000MAXMINAS} and lay pheromone in proportion to the change 
  in cost
  \item Repeat steps 1-3 until either a time limit is reached, a set number of 
  iterations has ran, or our pheromone matrix has converged to a constant value.
  \item Return our best solution gathered
\end{enumerate}

\section{Complexity Analysis}
\subsection*{Greedy Algorithm Complexity}
Assuming that our greedy algorithm will always find a solution, the greedy algorithm will go through
every city once and then stop. This will result in linear time and constant space, not including the
space supplied by the cities and edges. This however may be suboptimal due to longer edges potentially
leading to smaller edges down the line.

\subsection*{ACO Complexity}
As described in \cite{Li2015EfficiencyIO} complexity of each step of the ant colony 
optimization algorithm is as follows: 
\begin{itemize}
  \item Initialize the variables - $\mathcal{O}(n^2 + m$)
  \item A solution matrix is generated for each ant - $\mathcal{O}(n^2 + m$)
  \item Find solutions for each ant and populate pheromone matrix - $\mathcal{O}(n^2m$)
  \item Update pheromone matrix - $\mathcal{O}(n^2$)
  \item Return to step 2 if Imax has not been reached - $\mathcal{O}(nm)$
  \item Return result - $\mathcal{O}(1)$
\end{itemize}

In the preceding equations, $n$ is the number of nodes or cities connected to, and $m$ is the
number of ants used to process the distances between nodes. The space complexity of the algorithm 
is described as $\mathcal{O}(n^2)$ + $\mathcal{O}(nm)$. The initial distance matrix is attributed
to the $n^2$ and the $nm$ describes the space used where each ant's solution is stored before 
choosing the best result. The time complexity can be described as $\mathcal{O}(I_{max}n^2m)$. 
Since, without improving our convergence criterion, we don't have a guarantee that the algorithm
will terminate unless we fix a bound for the number of iterations, any upper bound on time complexity
has to be expressed as a function of $I_{max}$, as we have done. Each attribute between the maximum iterations, the amount of nodes, and number of ants contributes
to the runtime of this algorithm. Our solution to the TSP problem using ant colony optimization
differed from other solutions as we did not use a table to keep track of possible nodes to connect to.

\section{Results}
Our results, and the hyperparemters used to generate them, are described in Table \ref{results-table}. With this
choice of hyperparameters (which certainly did require some experimentation and fine-tuning), the
algorithm consistently outperforms our baselines. We did not run for more than 200 Cities because
at our fixed time limit of 10 minutes, it already was taking a ridiculous amount of time to get results.

\section{Discussion and Analysis}
We found that when searching for a solution for a lower number of nodes it was just as, if not more 
effective to use the Branch and Bound algorithm (which is guaranteed to find an optimal solution if
given enough time), but as the number of nodes increased, the Ant Colony Optimization algorithm greatly 
outperformed the others. This is due to the algorithm's ability to
test multiple examples against themselves each time and find an optimal solution. This means that
each algorithm used the maximum allotted time, but only our optimization algorithm was able to find 
a more optimal solution when approaching a larger number of nodes in a network. 

Pros: converges quickly towards a local solution while keeping variance in proportion to the number of iterations
Cons: Potential to get stuck in local minima with a smaller tau than sought after. This causes
the algorithm to effectively get stuck in an infinite loop until it runs into another local minima
that has a lower cost. (to counter this we looked at integrating a iteration counter which limited
the number of iterations after obtaining a solution)

\section{Future Work}
We realized after we had done our empirical analysis that we had missed an
aspect of the ACO algorithm which we likely should have included---something that
\cite{Li2015EfficiencyIO} calls a ``tabu table'', and that was described in a more
cursory fashion by \cite{Sttzle2000MAXMINAS} as the list of candidate cities which
ants primarily focus on traveling to at each decision point. For example, suppose
ant $k$ is currently located at city $i$. The list of candidate cities that the ant
considers traveling to from city $i$ is the subset of the $c$ cities closes to $i$,
denoted $\mathcal{N}_i^k$, which ant $k$ has not yet visited. The parameter $c$ would
be set manually beforehand (in the two papers we read, they both used $c=16$), and 
this would thus greatly restrict the space of possibilities over which we would compute
the probability distribution at each decision point. In the case where the ant has already
visited all $c$ cities closest to $i$, then the ant's choice would become more deterministic,
and it would just travel to the city with the lowest cost that it has not yet visited. Clearly, this
process would speed up the computation of each step, and we believe it could lead to improved
results.

Just as important of a change that we would like to make is improving our criterion for convergence.
We did not specify a fixed number of iterations for our algorithm, but instead tried to use
the notion of convergence described in \cite{Sttzle2000MAXMINAS}. However, this only worked in about
50-60\% of the cases that we tried---the algorithm would instead often run to the time limit. We 
believe that further investigation would tell us why this is the case, and by fixing those parameters
we could help the algorithm converge earlier.

Finally, with more time, we would like to tune our hyperparameters more thoroughly and carefully,
and identify in what types of regimes they work best. We would also like to explore the
literature on this topic more deeply, since many more people, far smarter than us, have
been researching in this space for far longer. We acknowledge that we have only scratched the
surface, but have learned a lot in the process.

% \csvautotabular{table.csv}

%% ---------------------------
%% Bibliography and postamble
%% ---------------------------
%%%%%%%%%%%%%%%%%%%%%%%%%%%%%%%%%%%%%%%%%%%%%%%%%%%%%%%%%%%%%%%%%%%%%%%%%%
%% This file was autogenerated by PaperShell v2.6.1 on 2023-04-18 07:28:33
%% https://github.com/sylvainhalle/PaperShell
%% DO NOT EDIT!
%%%%%%%%%%%%%%%%%%%%%%%%%%%%%%%%%%%%%%%%%%%%%%%%%%%%%%%%%%%%%%%%%%%%%%%%%%
\bibliographystyle{abbrv}

\bibliography{paper}
%%%%%%%%%%%%%%%%%%%%%%%%%%%%%%%%%%%%%%%%%%%%%%%%%%%%%%%%%%%%%%%%%%%%%%%%%%
%% This file was autogenerated by PaperShell v2.6.1 on 2023-04-18 07:28:33
%% https://github.com/sylvainhalle/PaperShell
%% DO NOT EDIT!
%%%%%%%%%%%%%%%%%%%%%%%%%%%%%%%%%%%%%%%%%%%%%%%%%%%%%%%%%%%%%%%%%%%%%%%%%%
%% ---------------------------
%% If there is anything you would like to insert *after* the bibliography
%% (and *before* the \end{document} instruction), place it here
%% ---------------------------

% \csvautotabular{table.csv}

\begin{table*}[]
    \centering
    \begin{tabular}{|c|cc|ccc|ccc|ccc|}
    \hline
                         & \multicolumn{2}{c|}{\textbf{Random}}                                                 & \multicolumn{3}{c|}{\textbf{Greedy}}                                                                                            & \multicolumn{3}{c|}{\textbf{Branch \& Bound}}                                                                                   & \multicolumn{3}{c|}{\textbf{Fancy}}                                                                        \\ \hline
    \textbf{\#   Cities} & \multicolumn{1}{l|}{\textbf{Time (sec)}} & \multicolumn{1}{l|}{\textbf{Length}} & \multicolumn{1}{l|}{\textbf{Time (s)}} & \multicolumn{1}{l|}{\textbf{Length}} & \multicolumn{1}{l|}{\textbf{\% of Random}} & \multicolumn{1}{l|}{\textbf{Time (s)}} & \multicolumn{1}{l|}{\textbf{Length}} & \multicolumn{1}{l|}{\textbf{\% of Greedy}} & \multicolumn{1}{l|}{\textbf{Time (s)}} & \multicolumn{1}{l|}{\textbf{Length}} & \textbf{\% of Greedy} \\ \hline
    30                   & 0.04                                     & 37303                                     & 0.0                                    & 18005                                     & 0.48                                       & 600.0                                  & 13063                                     & 0.73                                       & 291.2                                  & \textbf{12632}                            & 0.7                   \\
    35                   & 0.14                                     & 44131                                     & 0.0                                    & 19258                                     & 0.44                                       & 600.0                                  & 15676                                     & 0.81                                       & 301.38                                 & \textbf{14847}                            & 0.77                  \\
    43                   & 1.64                                     & 51050                                     & 0.01                                   & 21878                                     & 0.43                                       & 600.0                                  & 18189                                     & 0.83                                       & 449.71                                 & \textbf{16664}                            & 0.76                  \\
    53                   & 8.7                                      & 70828                                     & 0.01                                   & 26020                                     & 0.37                                       & 600.0                                  & 20884                                     & 0.8                                        & 600.05                                 & \textbf{18746}                            & 0.72                  \\
    54                   & 23.23                                    & 73954                                     & 0.0                                    & 27199                                     & 0.37                                       & 600.0                                  & 23018                                     & 0.85                                       & 600.05                                 & \textbf{19850}                            & 0.73                  \\
    60                   & 67.81                                    & 76344                                     & 0.0                                    & 27689                                     & 0.36                                       & 600.0                                  & 23365                                     & 0.84                                       & 535.8                                  & \textbf{20979}                            & 0.76                  \\
    62                   & 292.16                                   & 81308                                     & 0.01                                   & 27890                                     & 0.34                                       & 600.0                                  & 23568                                     & 0.85                                       & 600.06                                 & \textbf{21548}                            & 0.77                  \\
    80                   & 225.99                                   & 114203                                    & 0.01                                   & 34080                                     & 0.3                                        & 600.01                                 & 29512                                     & 0.87                                       & 600.13                                 & \textbf{27098}                            & 0.8                   \\
    92                   & TB                                       & TB                                        & 0.02                                   & 35628                                     & NA                                         & 600.01                                 & 30491                                     & 0.86                                       & 600.19                                 & \textbf{28239}                            & 0.79                  \\
    99                   & TB                                       & TB                                        & 0.03                                   & 39839                                     & NA                                         & 600.01                                 & 32532                                     & 0.82                                       & 600.08                                 & \textbf{29353}                            & 0.74                  \\
    100                  & TB                                       & TB                                        & 0.01                                   & 37903                                     & NA                                         & 600.01                                 & 33491                                     & 0.88                                       & 600.11                                 & \textbf{30426}                            & 0.8                   \\
    103                  & TB                                       & TB                                        & 0.01                                   & 38287                                     & NA                                         & 600.01                                 & 33137                                     & 0.87                                       & 600.18                                 & \textbf{30939}                            & 0.81                  \\
    116                  & TB                                       & TB                                        & 0.02                                   & 42554                                     & NA                                         & 600.01                                 & 36466                                     & 0.86                                       & 600.2                                  & \textbf{33391}                            & 0.78                  \\
    177                  & TB                                       & TB                                        & 0.04                                   & 53667                                     & NA                                         & 600.03                                 & 47253                                     & 0.88                                       & 600.48                                 & \textbf{45640}                            & 0.85                  \\
    189                  & TB                                       & TB                                        & 0.06                                   & 56303                                     & NA                                         & 600.03                                 & 51132                                     & 0.91                                       & 600.5                                  & \textbf{48098}                            & 0.85                  \\
    200                  & TB                                       & TB                                        & 0.09                                   & 58379                                     & NA                                         & 600.02                                 & 50866                                     & 0.87                                       & 600.5                                  & \textbf{50581}                            & 0.87                  \\ \hline
    \end{tabular}
    \caption{\label{results-table}Average empirical results over 5 iterations 
    of each value of $n$, obtained when running up to a 10-minute time limit. At each
    iteration, 10 ants were used with $\alpha=1, \beta=2, p_{best}=1$, and $\rho=.97$, }
\end{table*}

% Nothing



\end{document}

%% :folding=explicit:wrap=soft:mode=latex: