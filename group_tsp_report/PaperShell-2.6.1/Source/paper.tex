% !TEX root = paper.tex

%% ***************************************************************************
%% My paper
%%
%% Authors: Emmett Brown, Marty McFly, Biff Tannen
%%
%% NOTE: this file will not compile until you called the script
%% generate-preamble.php once. See the file Readme.md to understand what
%% to do.
%%
%% This paper is an instance of the PaperShell template. For more
%% information, please visit https://github.com/sylvainhalle/PaperShell
%% ***************************************************************************
%% ---------------------------
%% Author preamble. The contents of this file change depending on the
%% paper class you choose.
%% ---------------------------
  
%%%%%%%%%%%%%%%%%%%%%%%%%%%%%%%%%%%%%%%%%%%%%%%%%%%%%%%%%%%%%%%%%%%%%%%%%%
%% This file was autogenerated by PaperShell v2.6.1 on 2023-04-18 07:28:33
%% https://github.com/sylvainhalle/PaperShell
%% DO NOT EDIT!
%%%%%%%%%%%%%%%%%%%%%%%%%%%%%%%%%%%%%%%%%%%%%%%%%%%%%%%%%%%%%%%%%%%%%%%%%%
\documentclass[journal]{IEEEtran}
%FOO
% Usual packages
\usepackage[utf8]{inputenc}      % UTF-8 input encoding
\usepackage[T1]{fontenc}         % Type1 fonts
\usepackage[english]{babel}      % Hyphenation
\usepackage{graphicx}            % Import graphics
\usepackage{cite}                % Better handling of citations
\usepackage[scaled]{helvet} % Scale Helvetica
\usepackage[bookmarks=false]{hyperref}            % Better handling of references in PDFs
\usepackage{comment}             % To comment out blocks of text

\usepackage{amssymb}


% Title
\title{An Exploration of Ant Colony Optimization in the Travelling Salesperson Problem}

% Authors and affiliations
\author{%
\IEEEauthorblockN{{Braden} {Webb}%
}
\\
\IEEEauthorblockA{%
\textit{Mathematics}\\
%
}

\IEEEauthorblockN{{Caden} {Brinkman}%
}
\\
\IEEEauthorblockA{%
\textit{Computer Engineering}\\
%
}

\IEEEauthorblockN{{Gavin} {Bolin}%
}
\\
\IEEEauthorblockA{%
\textit{Computer Science}\\
%
}

%
}

\input{includes.tex}

\usepackage{newtxtext,newtxmath} % Times with math support. Must be placed here to avoid clash if amsthm is loaded in includes
\usepackage{microtype}       % Better handling of typo
%% Default path for graphics
\graphicspath{{fig/}}

\begin{document}

\maketitle
\begin{abstract}
%% ----------------------
%% Write your abstract here. Do not enclose it in an "abstract"
%% environment.
%% ----------------------
The traveling salesman problem has been the main focus of computer science researchers, economists and businessmen. If optimally implemented, the traveling salesman problem can result in decreased travel time and decreased loop times. This will allow the public to enjoy more time on vacation instead of using that time traveling and will allow businesses to minimize costs of daily routes traveled throughout the world. In our implementation of this famous algorithm we mimicked the routing of ants to determine where the optimal route will be. 
%% :folding=explicit:wrap=soft:mode=latex:

\end{abstract}

% Fixing bug in the definition of \markboth in IEEEtran class
% See http://tex.stackexchange.com/a/88864
\makeatletter
\let\l@ENGLISH\l@english
\makeatother


%% ---------------------------
%% If you wish to include additional packages, define new environments or
%% new commands, put them in the file includes.tex
%%
%% Write your abstract in the file abstract.tex.
%% ---------------------------

%% ---------------------------
%% Introduction
%% ---------------------------
\section{Introduction} %% {{{

We chose to implement the ant colony algorithm because of its use of randomness and probability which allow convergence to an optimal solution without getting rid of diversity in the population. It was similar to a Genetic Algorithm approach but used probability at every step instead of probability between switching confirmed paths.

%% }}} --- Section

%% ---------------------------
%% A section
%% ---------------------------
\section{Algorithm Explanation} %% {{{

\subsection*{Greedy}
Our Greedy algorithm simply takes the smallest edge cost branching out of the current node we are at until all nodes are visited and we make a complete loop.

\subsection*{Ant Colony Optimization Algorithm (ACO)}
The ant colony algorithm \cite{Skinderowicz_2022} mimics ants and their use of pheromones in order to locate potential food sources. This is done by detecting the concentration of each pheromone for each path and choosing based on probability the path with more pheromone. The pheromone for each path decreases as ‘time’ goes on.
Our implementation follows 5 steps:

\begin{enumerate}
  \item Initialize a population of n ants
  \item Allow the ants to choose paths with more preference on shorter paths and with more preference on paths with a higher concentration of pheromones
  \item For ants that achieved a valid route, choose the one that had the lowest cost and lay pheromone in proportion to the change in cost
  \item Repeat steps 1-3 until the probability that our best route is optimal is sufficiently higher than a set probability.
  \item Return our best solution gathered
\end{enumerate}

\section{Complexity Analysis}
\subsection*{Greedy Algorithm Complexity}
Assuming that our greedy algorithm will always find a solution, the greedy algorithm will go through
every city once and then stop. This will result in linear time and constant space, not including the
space supplied by the cities and edges. This however may be suboptimal due to longer edges potentially
leading to smaller edges down the line.

\subsection*{ACO Complexity}
The complexity of each step of the ant colony optimization algorithm is as follows: 
\begin{itemize}
  \item Initialize the variables - $\mathcal{O}(n^2 + m$)
  \item A solution matrix is generated for each ant - $\mathcal{O}(n^2 + m$)
  \item Find solutions for each ant and populate pheromone matrix - $\mathcal{O}(n^2m$)
  \item Update pheromone matrix - $\mathcal{O}(n^2$)
  \item Return to step 2 if Imax has not been reached - $\mathcal{O}(nm)$
  \item Return result - $\mathcal{O}(1)$
\end{itemize}

In the following equations, $n$ is the number of nodes or cities connected to, and $m$ is the
number of ants used to process the distances between nodes. The space complexity of the algorithm 
is described as $\mathcal{O}(n^2)$ + $\mathcal{O}(nm)$. The initial distance matrix is attributed
to the $n^2$ and the $nm$ describes the space used where each ant's solution is stored before 
choosing the best result. The time complexity can be described as $\mathcal{O}(I_{max}n^2m)$. 
Each attribute between the maximum iterations, the amount of nodes, and number of ants contributes
to the runtime of this algorithm. Our solution to the TSP problem using ant colony optimization
differed from other solutions as we did not use a table to keep track of possible nodes to connect to.

\section{Results}
Please refer to Appendix 1 for results.

\section{Discussion and Analysis}
We found that when searching for a solution for a lower number of nodes it was more effective to
use the Greedy or Branch and Bound algorithms, but as the number of nodes increased, the Ant Colony
Optimization algorithm greatly outperformed the others. This is due to the algorithm's ability to
test multiple examples against themselves each time and find an optimal solution. This means that
each algorithm used the maximum allotted time, but only our optimization algorithm was able to find 
a more optimal solution when approaching a larger number of nodes in a network. 

Pros: converges quickly towards a local solution while keeping variance in proportion to the number of iterations
Cons: * Potential to get stuck in local minima with a smaller tau than sought after. This causes
the algorithm to effectively get stuck in an infinite loop until it runs into another local minima
that has a lower cost. (to counter this we looked at integrating a iteration counter which limited
the number of iterations after obtaining a solution)

\section{Future Work}
add some stuff here

%% ---------------------------
%% Bibliography and postamble
%% ---------------------------
%%%%%%%%%%%%%%%%%%%%%%%%%%%%%%%%%%%%%%%%%%%%%%%%%%%%%%%%%%%%%%%%%%%%%%%%%%
%% This file was autogenerated by PaperShell v2.6.1 on 2023-04-18 07:28:33
%% https://github.com/sylvainhalle/PaperShell
%% DO NOT EDIT!
%%%%%%%%%%%%%%%%%%%%%%%%%%%%%%%%%%%%%%%%%%%%%%%%%%%%%%%%%%%%%%%%%%%%%%%%%%
\bibliographystyle{abbrv}

\bibliography{paper}
\input{postamble.inc.tex}

\end{document}

%% :folding=explicit:wrap=soft:mode=latex: